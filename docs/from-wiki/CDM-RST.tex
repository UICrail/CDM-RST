% Options for packages loaded elsewhere
\PassOptionsToPackage{unicode}{hyperref}
\PassOptionsToPackage{hyphens}{url}
\PassOptionsToPackage{dvipsnames,svgnames,x11names}{xcolor}
%
\documentclass[
  11pt,
]{article}
\usepackage{amsmath,amssymb}
\usepackage{setspace}
\usepackage{iftex}
\ifPDFTeX
  \usepackage[T1]{fontenc}
  \usepackage[utf8]{inputenc}
  \usepackage{textcomp} % provide euro and other symbols
\else % if luatex or xetex
  \usepackage{unicode-math} % this also loads fontspec
  \defaultfontfeatures{Scale=MatchLowercase}
  \defaultfontfeatures[\rmfamily]{Ligatures=TeX,Scale=1}
\fi
\usepackage{lmodern}
\ifPDFTeX\else
  % xetex/luatex font selection
\fi
% Use upquote if available, for straight quotes in verbatim environments
\IfFileExists{upquote.sty}{\usepackage{upquote}}{}
\IfFileExists{microtype.sty}{% use microtype if available
  \usepackage[]{microtype}
  \UseMicrotypeSet[protrusion]{basicmath} % disable protrusion for tt fonts
}{}
\makeatletter
\@ifundefined{KOMAClassName}{% if non-KOMA class
  \IfFileExists{parskip.sty}{%
    \usepackage{parskip}
  }{% else
    \setlength{\parindent}{0pt}
    \setlength{\parskip}{6pt plus 2pt minus 1pt}}
}{% if KOMA class
  \KOMAoptions{parskip=half}}
\makeatother
\usepackage{xcolor}
\usepackage[margin=1in]{geometry}
\usepackage{graphicx}
\makeatletter
\def\maxwidth{\ifdim\Gin@nat@width>\linewidth\linewidth\else\Gin@nat@width\fi}
\def\maxheight{\ifdim\Gin@nat@height>\textheight\textheight\else\Gin@nat@height\fi}
\makeatother
% Scale images if necessary, so that they will not overflow the page
% margins by default, and it is still possible to overwrite the defaults
% using explicit options in \includegraphics[width, height, ...]{}
\setkeys{Gin}{width=\maxwidth,height=\maxheight,keepaspectratio}
% Set default figure placement to htbp
\makeatletter
\def\fps@figure{htbp}
\makeatother
\setlength{\emergencystretch}{3em} % prevent overfull lines
\providecommand{\tightlist}{%
  \setlength{\itemsep}{0pt}\setlength{\parskip}{0pt}}
\setcounter{secnumdepth}{5}
\ifLuaTeX
  \usepackage{selnolig}  % disable illegal ligatures
\fi
\IfFileExists{bookmark.sty}{\usepackage{bookmark}}{\usepackage{hyperref}}
\IfFileExists{xurl.sty}{\usepackage{xurl}}{} % add URL line breaks if available
\urlstyle{same}
\hypersetup{
  pdftitle={CDM-RST Wiki Documentation},
  colorlinks=true,
  linkcolor={blue},
  filecolor={Maroon},
  citecolor={Blue},
  urlcolor={blue},
  pdfcreator={LaTeX via pandoc}}

\title{CDM-RST Wiki Documentation}
\author{}
\date{}

\begin{document}
\maketitle

{
\hypersetup{linkcolor=}
\setcounter{tocdepth}{3}
\tableofcontents
}
\setstretch{1.2}
\hypertarget{cdm-rst-compiled-wiki-pages}{%
\section{CDM-RST compiled wiki
pages}\label{cdm-rst-compiled-wiki-pages}}

\emph{Self-contained version with local images}

\hypertarget{version}{%
\subsection{Version}\label{version}}

This document was generated on 2025-11-19 15:01:42 UTC

\begin{center}\rule{0.5\linewidth}{0.5pt}\end{center}

\hypertarget{rolling-stock-consist}{%
\section{Rolling stock consist}\label{rolling-stock-consist}}

\hypertarget{purpose}{%
\subsection{Purpose}\label{purpose}}

Represent rolling stock vehicles and formations, their parthood
relationships, and the order and orientation of vehicles in a formation.

\hypertarget{vehicles-and-formations}{%
\subsection{Vehicles and formations}\label{vehicles-and-formations}}

\hypertarget{purpose-1}{%
\subsubsection{Purpose}\label{purpose-1}}

Simple distinction between a vehicle and a formation. A formation
consists of vehicles or other formations. A formation that cannot be
broken down further is called a ``vehicle''. The old debate is whether
the breakdown can, or cannot be performed in operations. ``It depends'',
not necessarily on the rolling stock, so the rolling stock ontology is
neutral about it.

The main statement, under the ontology, is: if a piece of rolling stock
cannot be broken down into other pieces of rolling stock, then it is a
vehicle.

A TGV trainset or ICE3 for instance have trailers, but these will not be
handled separately in operations (ICE case) and cannot even rest on
rails (TGV trailers rest on 0 to 2 bogies), but are still considered
``vehicles'' by many, and for sure they cannot be further broken down
into smaller pieces of rolling stock - only parts (carbodies, bogies,
HVAC units, seats\ldots).

\hypertarget{diagram}{%
\subsubsection{Diagram}\label{diagram}}

The diagram illustrates a composite pattern (known from UML) which
expresses that a formation may contain vehicles or other formations.

The flattened black hexagon is the GRAPHOL symbol for disjoint unions:
hence the set of all rolling stock contains two disjoint subsets,
vehicles and formations. A vehicle cannot be a formation, and
conversely. A formation X can however contain a single vehicle Y, but X
and Y remain distinct objects.

\begin{figure}
\centering
\includegraphics{images/cons_001 - Vehicles and Formations.png}
\caption{GRAPHOL diagram, formations}
\end{figure}

Property ``part of formation'' is functional (and it inverse ``includes
Rolling stock'' is inverse functional): this means that a piece of
rolling stock cannot be part of two formations. For reference, an OWL
``functional property'' has at most one object.

\hypertarget{comments}{%
\subsubsection{Comments}\label{comments}}

\hypertarget{more-details-about-the-representation-of-properties-in-graphol}{%
\paragraph{More details about the representation of properties in
GRAPHOL}\label{more-details-about-the-representation-of-properties-in-graphol}}

The GRAPHOL diagram (and the OWL2 ontology) also expresses that a
formation includes \emph{at least} one piece of rolling stock. The
double arrow (equivalence) between class Formation and the small white
box means exactly that. In detail:

\begin{itemize}
\tightlist
\item
  property ``includes Rolling stock'' has an extension, which is the set
  of pairs (a formation, a rolling stock) for which the property holds.
\item
  the white box represents the set of formations for which the property
  holds
\item
  the black box represents the set of rolling stock for which the
  property holds
\item
  the double arrow means mutual inclusion, i.e.~equivalence, so the
  extension of formation is equivalent to the set of formations that
  includes some (existential quantifier = at least one) rolling stock.
\item
  this double arrow entails ``every formation includes at least one
  rolling stock''.
\item
  please note that the RDFS domain semantics are different:
  \texttt{cons:includesRollingStock\ rdfs:domain\ cons:Formation} means:

  \begin{itemize}
  \tightlist
  \item
    ``a formation \emph{may include} rolling stock''
  \item
    the corresponding GRAPHOL representation would show a single arrow
    pointing from the white box into the Formation box (Formation is a
    super-set of the things that include rolling stock).
  \end{itemize}
\end{itemize}

\hypertarget{about-handling-of-constraints}{%
\paragraph{About handling of
constraints}\label{about-handling-of-constraints}}

We generally use OWL2 constraints when they also carry some semantics.
Here, having a container (formation) without contents (vehicles or other
formations) is pretty much meaningless.

Constraints only dealing with data correctness are better represented as
SHACL shapes. These can be checked, but do not interact with the
semantics of the data representation.

In some cases, we move to SHACL such constraints that are semantically
relevant but do not match the self-imposed limitations on OWL2 usage.
For instance, an ETCS Balise Group consists of at most 7 balises, but
the qualified cardinality restriction ``at most 7'' cannot be expressed
in the OWL2 RL profile; the OWL2 DL profile at least is required.

\hypertarget{extremities}{%
\subsection{Extremities}\label{extremities}}

\hypertarget{purpose-2}{%
\subsubsection{Purpose}\label{purpose-2}}

All rolling stock is oriented on the drawing board. Extremities are
usually designated ``extremity 1'', ``extremity 2'', and this
distinction helps differentiating driving cabs on a locomotive, the
placement of axles or loads, etc. independently from other information
(e.g.~running direction). Rolling stock orientation is conventional and
may rest on standards: see for instance EN 13775-1.

\hypertarget{diagram-1}{%
\subsubsection{Diagram}\label{diagram-1}}

The diagram illustrates two properties: hasExtremity (``a rolling stock
has two extremities'') and facesExtremity (on coupling, an extremity of
a rolling stock faces the extremity of another rolling stock).

\begin{figure}
\centering
\includegraphics{images/cons_002 - Extremities.png}
\caption{diagram: extremities}
\end{figure}

An extremity belongs to one wagon (it cannot be shared with another
one), so ``hasExtremity'' is inverse functional.

Extremities of two adjacent wagons may face each other: this is
expressed by ``faces extremity'', a symmetric property (if X1 faces Y2,
then Y2 faces X1).

Once extremities are facing each other, one may consider coupling them;
if they are coupled, they are obviously facing each other. This is
exactly the meaning of ``coupled to'' being a subproperty of ``faces
extremity''.

\hypertarget{comments-1}{%
\subsubsection{Comments}\label{comments-1}}

\hypertarget{about-cardinalities-subproperties-and-disjoint-properties}{%
\paragraph{About cardinalities, subproperties, and disjoint
properties}\label{about-cardinalities-subproperties-and-disjoint-properties}}

Property hasExtremity should have a cardinality of exactly 2 (not
available in OWL2 RL) and this cardinality is of little use anyway,
since we need to identify extremity 1 and extremity 2 that will orient
the rolling stock. These extremities are singled out by sub-properties
of hasExtremity. These sub-properties are disjoint: by essence,
extremity 2 of a wagon cannot be the same as extremity 1 of the same
wagon. This is expressed by the ``not'' flattened hexagon in GRAPHOL,
that tells that the set of (rolling stock, extremity) pairs satisfying
``extremity\_2'' does not overlap the set of pairs satisfying
``extremity\_1''. In set theory, Y does not overlap X iff Y is a subset
of the complement of X (as the complement of X is the largest set not
overlapping X).

\hypertarget{formations-as-ordered-sets}{%
\subsection{Formations as ordered
sets}\label{formations-as-ordered-sets}}

\hypertarget{purpose-3}{%
\subsubsection{Purpose}\label{purpose-3}}

Sequences, vectors, arrays are not part of OWL2 syntax elements. To
express order, one must use some List ontology expressed in OWL2. Here,
we use the same List ontology also used by IfcOwl. It rests on the
``linked list'' paradigm, with each element of the list pointing to the
next.

The list is invariably closed by an instance of EmptyList, a special
class provided by the List ontology. This ensures that the list has
actually come to an end.

\emph{Note: given the underlying ``Open World Assumption'', we do not
assume that data are complete, we only accept \textbf{positive
information telling that it is complete}. Here, the fact that a list
element does not point to any next element has no meaning. After all,
the next element could simply be unknown and get discovered later. But
if the list element points to an ``empty list'', this positively tells
that the list has come to an end.}

\hypertarget{diagram-2}{%
\subsubsection{Diagram}\label{diagram-2}}

The first diagram showed that a formation includes some (= at least one)
rolling stock. The diagram below expresses, on the left side, that the
included rolling stock can be the ``head rolling stock'', in which case
all other information can be pulled from the head rolling stock that
must indicate which successor (``has next rolling stock'') it has, until
the closing element of the list is reached. The closing element is an
individual of class EmptyList, imported from the List ontology; it
cannot be confused with a real piece of rolling stock.

\begin{figure}
\centering
\includegraphics{images/cons_003 - Consist.png}
\caption{diagram: rolling stock consist}
\end{figure}

\emph{Reading the GRAPHOL diagram from ``Listed rolling stock'' towards
the right: every listed rolling stock must (double arrow between
``listed rolling stock'' and the white square) have as next rolling
stock at most one (functional property, signalled by double edge)
rolling stock xor (disjoint union, signalled by black flattened hexagon)
an empty list.}

The bottom of the diagram adds property ``front extremity'' that
designates which extremity is at the front of the rolling stock in the
formation.

To know what is at the front of the formation, you can use the property
chain (has head rolling stock)\emph{o}(front extremity).

Original page:
\href{https://github.com/UICrail/CDM-RST/wiki/01-\%E2\%80\%90-Rolling-stock-consist}{01-‐-Rolling-stock-consist.md}

\end{document}
